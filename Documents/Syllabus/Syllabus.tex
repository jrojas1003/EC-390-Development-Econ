 \documentclass[11pt]{article}

% ---------------------------------------------------
% Page Layout
% ---------------------------------------------------
\usepackage{geometry}   % Flexible page dimensions
\geometry{margin=1in, centering}

% ---------------------------------------------------
% Encoding and Fonts
% ---------------------------------------------------
\usepackage[utf8]{inputenc}   % Input encoding
\usepackage[T1]{fontenc}      % Output font encoding
\usepackage{lmodern}          % Latin Modern fonts (improves output)

% ---------------------------------------------------
% Tables and Columns
% ---------------------------------------------------
\usepackage{multicol}   % Multiple column environments
\usepackage{multirow}   % Multi-row cells in tables
\usepackage{booktabs}   % Better looking tables

% ---------------------------------------------------
% Graphics and Figures
% ---------------------------------------------------
\usepackage{graphicx}   % Include graphics
\usepackage{caption}    % Customization of captions

% ---------------------------------------------------
% Text Styling
% ---------------------------------------------------
\usepackage{color}      % Color support
\usepackage{soul}       % Highlighting, underlining, strikethrough, etc.

% ---------------------------------------------------
% Headers, Footers, and Symbols
% ---------------------------------------------------
\usepackage{fancyhdr}   % Custom headers/footers
\usepackage{lastpage}   % Reference last page in document
\usepackage{bbding}     % Special symbols (checkmarks, etc.)
\usepackage{pmboxdraw} % Box drawing characters

% ---------------------------------------------------
% Section Formatting
% ---------------------------------------------------
\usepackage{titlesec}   % Customize section titles

% ---------------------------------------------------
% Hyperlinks and URLs
% ---------------------------------------------------
\usepackage{url}        % Simple URL typesetting
\usepackage{hyperref}   % Clickable hyperlinks


\titleformat{\section}
  {\normalfont\large\bfseries\sffamily}{\thesection}{1em}{}

\titleformat{\subsection}
  {\normalfont\normalsize\bfseries\sffamily}{\thesection}{1em}{}
  
\hypersetup{
    colorlinks=true,
    linkcolor=blue,
    filecolor=magenta,      
    urlcolor=blue,
    citecolor=blue,
}

\pagenumbering{arabic}

\begin{document}

\fontfamily{lmss}\selectfont

\begin{center}
\textbf{\huge EC 390: Development Economics}\\ 
\vspace{0.1in}
University of Oregon\\ 
Department of Economics\\ 
\end{center}
\vspace{-.2in}
\begin{center}\begin{tabular}{llll}
\toprule
    \textbf{Instructor:} & \footnotesize{Jose Rojas-Fallas} & \textbf{Class Location:}  & {\footnotesize Gerlinger 302} \\ 
    \textbf{Email:} & \footnotesize{\href{mailto:jrojas2@uoregon.edu}{jrojas2@uoregon.edu}} & \textbf{Class Day/Time:} & \footnotesize{Mon \& Wed: 12:00 - 1:20 pm} \\
    \textbf{Office Hours:} & \footnotesize{Tues 09:00 - 11:00 am} & \textbf{Office:} & \footnotesize{PLC 525}\\
\bottomrule
\end{tabular}
\end{center}

\bigskip

\section*{COURSE SUMMARY}

\begin{center}
    \fbox{\fbox{\parbox{5.5in}{\centering
    "In some sense, development economics used to be at the centre of all of economics.
    The classical economists of the 17th, 18th, and early 19th centuries were all development economists, as they were usually writing about a developing country (...)" - Bardhan \& Udry \\

    "We experts don't care about rising gross domestic product for its own sake.
    We care because it betters the lot of the poor and reduces the proprtion of people who are poor.
    We care because richer people can eat more and buy more medicines for their babies." - W. Easterly
    }}}\end{center}

\subsection*{DESCRIPTION}
Development Economics is about problems facing billions of people in low- and middle-income countries.
It is also about enormous difference in living standards across the riches and poorest nations. 

Nearly half of 6.1 billion people living in the developing nations lack access to safe sweres and a third have no access to clean water. 
The richest 20 \% of the world population consumes 86 \% of all goods and services, while the poorest 20 \% consumes just 1.3 \%. 

The objective of this course is to provide some understanding into these stark, and occasionally not-so-obvious, problems being faced by a majority of the world. 

\subsection*{PREREQUISITES}
The prerequisite for this course is EC 201: Principles of Microeconomics. 

\section*{LEARNING OBJECTIVES}
\begin{itemize}
    \item Explain key theories and concepts in development economics, including poverty traps, human capital, institutions, and market failures
    \item \hl{Analyze the role of education, health, credit, trade, and governance in shaping economic development outcomes}
    \item Interpret and critically evaluate empirical evidence on development policies and interventions
    \item \hl{Apply economic models to real-world development challenges and assess their assumptions and limitations}
    \item \hl{Formulate a clear, researchable question in development economics and motivate its importance}
    \item Communicate complex economic ideas and policy debates clearly, both in writing and in presentations
\end{itemize}

\section*{LEARNING MATERIAL:}
\begin{itemize}
    % TALK ABOUT TEXTBOOK HERE
    \item \textbf{Textbook:} There is one \textbf{recommended} textbook for this course:
        \begin{enumerate}
            \item \textbf{Economic Development}, $12^{th}$ Edition, by M. Todaro \& S. Smith
        \end{enumerate}
    It is available at the \href{https://www.uoduckstore.com/book-search-results?crn=11737&term=202501}{Duck Store (ISBN: 9780133406788)}.
    Attending lecture is not a perfect substitute for reading and comprehending the text. 
    Similarly, only reading is not a perfect substitute for for attending lecture. 
    % TALK ABOUT ARTICLES HERE
    \item \textbf{Additional Readings:} In addition to the textbook, I may assign readings from peer-reviewed studies or news articles for classroom discussion. 
    I am responsible for providing any additional material to you.
    % TALK ABOUT LECTURE NOTES HERE
    \item \textbf{Lecture Notes:} Lectures will be complemented by slides.
    I will expand on the topics, statements, facts, images, and graphs included in the slides during lecture. 
    The slides will be made available at the start of the class in which they will be presented. 
    
    \emph{Note: These serve as a guiding outline which I will follow during lectures but are not necessarily the full material. They may help keep your notes organized.}
\end{itemize}

\bigskip 

\section*{ASSIGNMENTS AND GRADING}

\begin{table}[ht]
    \centering
    \begin{tabular}{c|l}
    \textbf{Assignment} & \textbf{Weight} \\
    \toprule
    Problem Sets $(\times 5)$  & \textbf{25 \%} \\
    Quizzes $(\times 4)$ & \textbf{10 \%} \\
    Midterm Exam  & \textbf{30 \%} \\
    Final Exam  & \textbf{30 \%} \\
    International Market Game  & \textbf{5 \%} \\
    \bottomrule
    \end{tabular}
    \label{Grade-Dist}
\end{table}

\bigskip 

\subsection*{PROBLEM SETS}
I will assign \textbf{4 Problem Sets} throughout the term. 
Problems will function as added practice for demonstrating an understanding of the course material. 

\begin{itemize}
    \item I will announce due dates in class within a reasonable period of time before they are due
    \item You will be turn in an \textbf{electronic copy} of each problem set on Canvas. 
    \textbf{Only PDF files will be accepted}.
\end{itemize}
Working in groups is encouraged as it will help you correct misunderstandings earlier on. 
Unless explicitly stated, \textbf{each student is required to write and submit their own independent answers}. 
Submitting duplicate work is subject to academic dishonesty concerns. 
If you work with others, \textbf{list their names at the top of your assignment}. 
Groups are expected to be of a reasonable size (\textbf{5 or less individuals}). 
If you submit the same work as other and fail to list collaborators, you will receive a score of zero for that problem set. 

\subsection*{QUIZZES}
I will assign \textbf{4 Quizzes} throughout the term. 
Quizzes will function as a test of your ability to use the course theory to real world contexts. 
They will be open-ended questions where you are tasked with arguing in favor or against a prompt, give your opinion on a circumstance, or come up with your own research idea. 
\textbf{There is no single correct answer}, but rather the main focus is on developing your use of logic/reasoning in your arguments. 

\begin{itemize}
    \item I will announce due dates in class within a reasonable period of time before they are due
    \item These will be completed on Canvas through the Quiz feature
\end{itemize}
Quizzes are completely individual and no group work is allowed. 
Submitting duplicate work is subject to academic dishonesty concerns. 
Answers should be no longer than one or two paragraphs. 

\subsection*{EXAM}
All exams are taken in-person at the assigned times. 
Any form of accommodation is only valid if informed through the Accessible Education Center (AEC).

There are \textbf{1 total exam -- a Midterm}. 

The \textbf{Midterm is planned for Wednesday, October 29th (Week 05)}. 
This is subject to change conditional on the pace of the class.
Any change will be communicated in an appropriate amount of time. 

There are no make-up exams. 
Nor will there be any student able to take an exam earlier or later than the scheduled time. 
Exams will consist of a mix of multiple-choice, short-answer, and multipart questions. 
During the exam you are allowed a set of writing utensils and a non-programmable calculator. 

\subsection*{FINAL PROJECT}
The final project for this course is a research proposal that engages with a topic in development economics. 
Students will identify a specific question related to development economics, motivate why the question is important, and \hl{propose a strategy to address it using theory, data, or policy analysis.}

\textbf{The focus is not on completing a full research paper}, but on demostrating the ability to clearly articulate a researchable question, explain its relevance, and outline a feasible approach to answering it.

Throughout the term, there will be intermediate steps to help guide your final project:

\begin{itemize}
    \item Topic Selection \hl{(Week X)}: Submit a brief (1-2 paragraph) description of your proposed research question, including why you believe it is relevant to development economics.
    \item Annotated Bibliography \hl{(Week X + 2)}: Provide \textbf{at least THREE peer-reviewed articles} related to your topic, each with a brief summary of how it informs your topic. 
    \item Outline \hl{(Week X + 4)}: Submit a draft outline or partial slides presentation summarizing your question, motivation, and proposed approach. 
    \item Final Presentation \hl{(Week 10)}: Deliver a 10-minute presentation of your research project. 
    \item Final Slides Submission \hl{(Finals Week)}: Submit your polished slides.
\end{itemize}

\noindent More details and information will be provided throughout the term. 

\newpage

\section*{COURSE POLICIES \& RESOURCES}

\subsection*{ACADEMIC INTEGRITY AND HONESTY}
Academic dishonesty will not be tolerated.
This includes any form of cheating or plagiarism.
Please familiarize yourself with the \href{https://policies.uoregon.edu/vol-3-administration-student-affairs/ch-1-conduct/student-conduct-code}{Student Conduct Code}.
If there are any questions about whether an act constitutes academic misconduct, it is the students' obligation to clarify the question with the instructor before committing or attempting to commit the act.

\subsection*{ACCOMMODATIONS FOR DISABILITIES}
If you have a documented disability and anticipate needing accommodations in this course, please let me know as soon as possible.
If there are any aspects of the instruction or design of this course that result in barriers to your participation, please contact me -- your success and the success of your peers is most important. 

I encourage you to contact the \href{https://aec.uoregon.edu/}{Accessible Education Center (AEC)}. The AEC offers a wide range of support services including note-taking, testing services, sign language interpretation and adaptive technology.

\subsection*{LATE POLICY}
I will not accept late assignments after the due date. 
If you turn in a problem set or quiz on the due date but after the deadline, points will be deducted for lateness. 
If you turn in an assignment after the answer key is made public, you will receive a zero. 

\textbf{I do not give makeup assignments}. 
This blanket ban extends to exams. 
\hl{In extreme circumstances that lead you to miss the midterm exam, I will consider re-weighting your grade toward the final exam. 
\textbf{To qualify for re-weighting, you must notify me no later than two days after exam}.
\textbf{Consideration for this form of accomodation is entirely subjective}.} 

\subsection*{GRADE APPEALS}
You must submit any request for re-grading in writing (via email) within \textbf{three (3) business days} of the day grades are posted for the problem set or exam in question. 
Your request should include a cogent argument explaining why your response(s) warrant full credit. 

\subsection*{RESPECT FOR DIVERSITY}
You can expect to be treated with respect in this course.
Both students and the instructor enter with many identities, backgrounds, and beliefs.
Students of all racial identities, ethnicities, gender identities, gender expressions, national origins, religious affiliations, sexual orientations, immigration status, ability and other non-visible differences belong in and contribute to this class and the discipline.

The UO Economics Department welcomes and respects diverse experiences, perspectives, and approaches. 
Both nationwide and at the University of Oregon, disproportionately few women and members of historically underrepresented racial and ethnic minority groups graduate with degrees in economics. 
All class participants are expected to communicate with respect and to avoid behaviors or contributions that undermine, demean, or marginalize others based on race, ethnicity, gender, sex, age, sexual orientation, religion, ability, or socioeconomic status.

Class rosters are provided to the instructor with students' legal names.
Please let me know if the name or pronouns we are provided for yourself are not accurate.
It is important to myself and others that you are addressed in your most preferred way.

\subsection*{ACADEMIC DISRUPTION}
In the event of a campus emergency that disrupts academic activities, course requirements, deadlines, and grading percentages are subject to change. 
Information about changes in this course will be communicated as soon as possible by email and on Canvas. 
Even though we will be meeting in-person every week, students should make sure to frequently log onto Canvas and read any announcements and/or access alternative assignments. 
Students are also expected to continue coursework as outlined in this syllabus or other instructions on Canvas. 

\vspace{1.5in}

\begin{table}[ht]
    \sffamily % Changes the font to correct font used in rest of text
    \centering
    \caption*{\sffamily \LARGE \textbf{TENTATIVE FALL TERM SCHEDULE}}
    \resizebox{\textwidth}{!}{%
    \begin{tabular}{l@{\hskip 0.5in} c c}  
    \toprule \toprule
        \textbf{Week} & \textbf{Content} & \textbf{Material} \\
    \midrule
        \textbf{Week 1} & Growth and Development & 1 and 2 \\
    \midrule
        \textbf{Week 2} & Theories of Growth and Development & 3 and 4 \\
    \midrule
        \textbf{Week 3} & Poverty and Inequality & 5 \\
    \midrule
        \textbf{Week 4} & Agriculture and Rural Development & 9 \\
    \midrule
        \textbf{Week 5} & Population & 6 \\
    \midrule
        \textbf{Week 6} & Urbanization and Rural-Urban Migration & 7 \\
    \midrule
        \textbf{Week 7} & Education and Health & 8 \\
    \midrule
        \textbf{Week 8} & Governance and Institutions & 11 \\
    \midrule
        \textbf{Week 9} & Foreign Aid, Trade, \& Globalization & 12 and 14 \\
    \midrule
        \textbf{Week 10} & Final Project Presentation & - \\
    \midrule
        \textbf{Finals Week} & Freedom & - \\
    \bottomrule \bottomrule
    \multicolumn{3}{c}{\footnotesize Topics are subject to change depending on class pace. The content will not.} \\
    \multicolumn{3}{c}{\footnotesize I will update dates as needed during the term.}
    \end{tabular}}
    \label{tab:schedule}
\end{table}

\end{document}