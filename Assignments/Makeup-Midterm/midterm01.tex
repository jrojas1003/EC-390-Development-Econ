%%%%%%%%%%%%%%%%%%%%%%%%%%%%%%%%%%%%%%%%%%%%%%%%%%%%
%                EC 380 MIDTERM TEMPLATE
%                  LaTeX Preamble
%%%%%%%%%%%%%%%%%%%%%%%%%%%%%%%%%%%%%%%%%%%%%%%%%%%%

\documentclass[12pt]{exam} % Exam document class with 12pt font

%%%%%%%%%%%%%%%%%%%%%%%%%%%%%%%%%%%%%%%%%%%%%%%%%%%%
%  FONT AND ENCODING
%%%%%%%%%%%%%%%%%%%%%%%%%%%%%%%%%%%%%%%%%%%%%%%%%%%%
\usepackage{libertine}           % Professional serif font (Linux Libertine)
\usepackage[utf8]{inputenc}      % Allows accented characters and symbols

%%%%%%%%%%%%%%%%%%%%%%%%%%%%%%%%%%%%%%%%%%%%%%%%%%%%
%  PAGE GEOMETRY AND LAYOUT
%%%%%%%%%%%%%%%%%%%%%%%%%%%%%%%%%%%%%%%%%%%%%%%%%%%%
\usepackage[margin=1in]{geometry} % Sets 1-inch margins
\pagestyle{headandfoot}           % Custom headers and footers (exam class feature)
\firstpageheadrule                % Horizontal line on first page header
\runningheadrule                  % Horizontal line on following pages

%%%%%%%%%%%%%%%%%%%%%%%%%%%%%%%%%%%%%%%%%%%%%%%%%%%%
%  MATH AND SYMBOLS
%%%%%%%%%%%%%%%%%%%%%%%%%%%%%%%%%%%%%%%%%%%%%%%%%%%%
\usepackage{amsmath, amssymb}     % AMS math environments and extra symbols
\usepackage{amsthm}               % Theorem, lemma, and proof environments
\usepackage{siunitx}              % Consistent formatting for numbers and units
\usepackage{cancel}               % Strike out math terms (e.g., unit cancellation)

%%%%%%%%%%%%%%%%%%%%%%%%%%%%%%%%%%%%%%%%%%%%%%%%%%%%
%  LAYOUT TOOLS AND LISTS
%%%%%%%%%%%%%%%%%%%%%%%%%%%%%%%%%%%%%%%%%%%%%%%%%%%%
\usepackage[shortlabels]{enumitem} % Custom list formatting (e.g., [a)], [i])
\usepackage{multicol}              % Enables multi-column layout (e.g., two-column questions)

%%%%%%%%%%%%%%%%%%%%%%%%%%%%%%%%%%%%%%%%%%%%%%%%%%%%
%  GRAPHICS AND VISUALS
%%%%%%%%%%%%%%%%%%%%%%%%%%%%%%%%%%%%%%%%%%%%%%%%%%%%
\usepackage{graphicx}              % Include external images
\usepackage{tikz}                  % Create vector graphics directly in LaTeX

%%%%%%%%%%%%%%%%%%%%%%%%%%%%%%%%%%%%%%%%%%%%%%%%%%%%
%  CODE AND TEXT HIGHLIGHTING
%%%%%%%%%%%%%%%%%%%%%%%%%%%%%%%%%%%%%%%%%%%%%%%%%%%%
\usepackage{listings}              % Code formatting and syntax highlighting
\usepackage{soul}                  % Text highlighting and underlining

%%%%%%%%%%%%%%%%%%%%%%%%%%%%%%%%%%%%%%%%%%%%%%%%%%%%
%  EXAM CLASS SETTINGS
%%%%%%%%%%%%%%%%%%%%%%%%%%%%%%%%%%%%%%%%%%%%%%%%%%%%
\printanswers                    % Uncomment to show answers
\CorrectChoiceEmphasis{\color{red}\bfseries\itshape} % Style for correct answers
\addpoints                         % Enable point tracking throughout exam
\bracketedpoints                   % Show point values in [brackets] instead of (parentheses)

%%%%%%%%%%%%%%%%%%%%%%%%%%%%%%%%%%%%%%%%%%%%%%%%%%%%
%  CUSTOM COMMANDS FOR COURSE INFORMATION
%%%%%%%%%%%%%%%%%%%%%%%%%%%%%%%%%%%%%%%%%%%%%%%%%%%%
\newcommand{\class}{\textbf{EC 390}}                  % Course name
\newcommand{\examnum}{\textbf{Midterm Exam}}   % Exam identifier
\newcommand{\examdate}{\textbf{October 29}}        % Exam date
\newcommand{\timelimit}{}                             % (Optional) Time limit placeholder

%%%%%%%%%%%%%%%%%%%%%%%%%%%%%%%%%%%%%%%%%%%%%%%%%%%%
%  HEADER AND FOOTER CONFIGURATION
%%%%%%%%%%%%%%%%%%%%%%%%%%%%%%%%%%%%%%%%%%%%%%%%%%%%
\firstpageheader{\class}{\examnum}{\examdate}
\runningheader{\class}{\examnum}{\examdate}

\firstpagefooter{Points earned: \makebox[1in]{\hrulefill} / 
\pointsonpage{\thepage} points}{}{\thepage\ of \numpages}

\runningfooter{Points earned: \makebox[1in]{\hrulefill} / 
\pointsonpage{\thepage} points}{}{\thepage\ of \numpages}

%%%%%%%%%%%%%%%%%%%%%%%%%%%%%%%%%%%%%%%%%%%%%%%%%%%%
%  TITLE INFORMATION
%%%%%%%%%%%%%%%%%%%%%%%%%%%%%%%%%%%%%%%%%%%%%%%%%%%%
\title{Make-Up Midterm Exam}
\author{EC 390 -- Development Economics}
\date{Fall 2025}

\begin{document}
\fontfamily{lmss}\selectfont
%-------------------------------------------------------------------------------
\begin{coverpages}

\maketitle

\begin{center}
\fbox{\fbox{\parbox{5.5in}{\centering
Give every question your best attempt. \\
Best of luck. \\
You got this!
}}}\end{center}
\vspace{2cm}
\makebox[0.6\textwidth]{Name:\enspace\hrulefill}
\makebox[0.35\textwidth]{95\#:\enspace\hrulefill}

\vspace{1cm}
\noindent
The maximum amount of points on this exam is 100 points. 
You have a total of 1h 20min (80 minutes) to complete the exam, unless otherwise noted. 
The only items allowed on your desk at any time are a pen and/or pencil, scratch paper, a 3x5 note card, and a calculator. 
Everything else must be stored in your bag underneath your desk. 
Any form of cheating will result on a zero on the exam.\\

\noindent There are three sections to be completed:

\begin{itemize}
    \item \textbf{Multiple Choice:} 5 Questions
    \item \textbf{Short Answer Questions:} 2 Questions
    \item \textbf{Multi-Part Analysis Questions:} 1 Question (5 parts)
\end{itemize}

\noindent Point totals and question specific instructions are listed for each section.
Please ask for clarification if a question is not clear to you.\\

\noindent The exam is a total of \numpages $\,$ pages. 
There are 8 pages of questions + 2 pages for scratch paper.

\noindent \textbf{Please verify you have all \numpages $\,$ in your exam. If you do not, let me know immediately.}

\end{coverpages}
%-------------------------------------------------------------------------------

%-------------------------------------------------------------------------------
\section*{Multiple Choice - 40 Points}
Circle or "X" the answer you think most correctly answers the following questions. 
If you mark a choice and would like to change it, \textbf{clearly indicate which one is your correct answer}. 
%-------------------------------------------------------------------------------
\begin{questions}

%%%%%%%%%%%%%%%%%%%%%%%%%%%%%%%%%%%%%%%%%%%%%%%%%%%%
%  QUESTION 01
%%%%%%%%%%%%%%%%%%%%%%%%%%%%%%%%%%%%%%%%%%%%%%%%%%%%
\question[4]
Country A reports their GDP to be \$8 Billion.
Residents earn \$2 Billion abroad, non-residents earn \$1.5 Billion domestically.
If domestic firms save \$3 Billion this year, what is the \textbf{Gross National Income (GNI)} of Country A?

\begin{choices}
    \choice \$14.5 Billion
    \choice \$ 11.5 Billion
    \CorrectChoice \$8.5 Billion
    \choice \$10 Billion
\end{choices}
\vspace*{\stretch{1}}

%%%%%%%%%%%%%%%%%%%%%%%%%%%%%%%%%%%%%%%%%%%%%%%%%%%%
%  QUESTION 02
%%%%%%%%%%%%%%%%%%%%%%%%%%%%%%%%%%%%%%%%%%%%%%%%%%%%
\question[4]
The ``Education Basket'' cointains: 1 textbook, 3 pens, 2 notebooks. 
Prices for \textbf{Richland} and \textbf{Poortown} are given below:

\begin{table}[htbp]
    \centering
    \begin{tabular}{|lcc|}
        \hline
        Item & Richland & Poortown\\
        \hline
        Textbook & 30 & 13 \\
        Pen & 2 & 1 \\
        Notebook & 5 & 3 \\
        Backpack & 50 & 42 \\
        \hline
    \end{tabular}
\end{table}

What is the \textbf{Purchasing Power Parity} between \textbf{Richland} and \textbf{Poortown} $(PPP_{Rich.,Poor.})$?

\begin{choices}
    \choice 37/17 = 2.17
    \choice 87/59 = 1.47
    \CorrectChoice 46/22 = 2.09
    \choice 22/46 = 0.47
\end{choices}
\vspace*{\stretch{1}}

%%%%%%%%%%%%%%%%%%%%%%%%%%%%%%%%%%%%%%%%%%%%%%%%%%%%
%  QUESTION 03
%%%%%%%%%%%%%%%%%%%%%%%%%%%%%%%%%%%%%%%%%%%%%%%%%%%%
\question[4]
Suppose that Chile follows the \textbf{Harrod-Domar} model of economic growth.
What will happen to the growth rate of Chile if an earthquake strikes the country and their \textbf{capital efficiency falls}, assuming the \textbf{savings rate} does not change?

\begin{choices}
    \choice Growth rate $\uparrow$, because it now takes more capital to produce one unit of output
    \CorrectChoice Growth rate $\downarrow$, because it now takes more capital to produce one unit of output
    \choice Growth rate does not change, because the savings rate does not change
    \choice Capital efficiency does not factor in the Harrod-Domar model
\end{choices}
\vspace*{\stretch{1}}

\newpage

%%%%%%%%%%%%%%%%%%%%%%%%%%%%%%%%%%%%%%%%%%%%%%%%%%%%
%  QUESTION 04
%%%%%%%%%%%%%%%%%%%%%%%%%%%%%%%%%%%%%%%%%%%%%%%%%%%%
\question[4]
\textbf{Conditional convergence} implies that:

\begin{choices}
    \choice The raito of income between poor and rich countries remains constant over time
    \choice All countries converge to identical $y^{*}$
    \choice Absolut convergence exists
    \CorrectChoice Countries converge conditional on country characteristics
\end{choices}
\vspace*{\stretch{1}}

%%%%%%%%%%%%%%%%%%%%%%%%%%%%%%%%%%%%%%%%%%%%%%%%%%%%
%  QUESTION 05
%%%%%%%%%%%%%%%%%%%%%%%%%%%%%%%%%%%%%%%%%%%%%%%%%%%%
\question[4]
Consider the \textbf{Two-Sector Lewis Model}. 
Assume that the agricultural (traditional) sector has surpulus labor, and the industrial (modern) sector reinvests all profits.
Suppose that \textbf{productivity in the agricultural sector rises dueto technological improvements}.

Which of the following statements best describes the dynamic effect of this productivity increase on the process of structural transformation?

\begin{choices}
    \CorrectChoice It slows the transfer of labor to the modern sector by raising the equilibrium subsistence wage, reducing the modern sector's profit rate
    \choice It accelerates the transfer of labor to the modern sector because rural wages remain fixed while agricultural output rises
    \choice It delays the Lewis turning point because higher agricultural productivity incrases profits in the modern sector, sustaining capital accumulation
    \choice It has no long-run effect on structural transformation since labor migration depends only on the savings rate in the modern sector
\end{choices}
\vspace*{\stretch{1}}

%%%%%%%%%%%%%%%%%%%%%%%%%%%%%%%%%%%%%%%%%%%%%%%%%%%%
%  QUESTION 06
%%%%%%%%%%%%%%%%%%%%%%%%%%%%%%%%%%%%%%%%%%%%%%%%%%%%
\question[4]
Which of the following statements is correct?

\begin{choices}
    \choice When there is perfect inequality in a country, the Lorenz Curve is a 45$^{\circ}$ line
    \choice Gini coefficient of 1 for a country indicates perfect equality
    \choice Gini coefficient of 0.5 for a country indicates perfect inequality
    \CorrectChoice None of the above
\end{choices}
\vspace*{\stretch{1}}

%%%%%%%%%%%%%%%%%%%%%%%%%%%%%%%%%%%%%%%%%%%%%%%%%%%%
%  QUESTION 07
%%%%%%%%%%%%%%%%%%%%%%%%%%%%%%%%%%%%%%%%%%%%%%%%%%%%
\question[4]
Lorelai values future payoffs more than Rory. 
This means:

\begin{choices}
    \choice Lorelai has a higher discount rate than Rory
    \CorrectChoice Lorelai has a lower discount rate than Rory
    \choice Lorelai and Rory have the same discount rate
    \choice Rory is more wealthy than Lorelai
\end{choices}
\vspace*{\stretch{1}}

\newpage
%%%%%%%%%%%%%%%%%%%%%%%%%%%%%%%%%%%%%%%%%%%%%%%%%%%%
%  QUESTION 08
%%%%%%%%%%%%%%%%%%%%%%%%%%%%%%%%%%%%%%%%%%%%%%%%%%%%
\question[4]
In a developing economy, multiple sectors require electricity to operate profitably. 
However, the power company can only cover fixed costs if enough firms demand electricity. 
Because no firm wants to invest until reliable power is available, the economy remains stagnant.
Which of the following government actions would most effectively address this \textbf{coordination failure}?

\begin{choices}
    \CorrectChoice Publicly invest in the power grid and guarantee initial energy supply constraints
    \choice Offer firms short-term credit at the market rate
    \choice Reduce income taxes for high-income households
    \choice Alter the savings rate through their central bank
\end{choices}
\vspace*{\stretch{1}}

%%%%%%%%%%%%%%%%%%%%%%%%%%%%%%%%%%%%%%%%%%%%%%%%%%%%
%  QUESTION 09
%%%%%%%%%%%%%%%%%%%%%%%%%%%%%%%%%%%%%%%%%%%%%%%%%%%%
\question[4]
In the article \emph{``To Do With the Price of Fish''}, access to cellphones led to price stabilization in the market for fish because

\begin{choices}
    \choice Cellphones led to a reduction in the number of fishermen, which increased the price of fish and led to higher profits
    \choice Cellphones allowed fishermen to sell directly to customers, rather than sell their fish to intermediaries
    \choice Fishermen sold their phones for better fishing equipment, which led to higher profits
    \CorrectChoice Cellphones allowed for fishermen to overcome a coordination failure
\end{choices}
\vspace*{\stretch{1}}

%%%%%%%%%%%%%%%%%%%%%%%%%%%%%%%%%%%%%%%%%%%%%%%%%%%%
%  QUESTION 10
%%%%%%%%%%%%%%%%%%%%%%%%%%%%%%%%%%%%%%%%%%%%%%%%%%%%
\question[4]
A country has abundant foreign savings and low inflation. 
However, \textbf{private investment remains very low}.
Data shows that: (1) Real interest rates are very high; (2) Firms report difficulty obtaining loans despite strong profitability; (3) Returns to education are low and unemployment among skilled workers is high.

According to the growth diagnostics approach, which of the following is most likely the \textbf{binding constraint} to growth?

\begin{choices}
    \choice Low human capital accumulation
    \choice Macroeconomic instability
    \CorrectChoice Poor access to financing
    \choice Low social returns to infrastructure
\end{choices}
\vspace*{\stretch{1}}

%-------------------------------------------------------------------------------
\newpage
\section*{Short Answer - 20 Points}
Answer the following questions to the best of your ability. 
For full credit, show all of your work and clearly indicate your final solution for each part by circling the answer.
%-------------------------------------------------------------------------------

%%%%%%%%%%%%%%%%%%%%%%%%%%%%%%%%%%%%%%%%%%%%%%%%%%%%
%  QUESTION 11
%%%%%%%%%%%%%%%%%%%%%%%%%%%%%%%%%%%%%%%%%%%%%%%%%%%%
\question

Consider the following graph

\begin{figure}[htbp!]
    \centering
    \includegraphics[width=0.85\textwidth]{images/multiple-equi.png}
\end{figure}

\begin{parts}
    \part[8]
    \textbf{Identify and label the equilibria points} on the graph. 
    In the space below, \textbf{categorize each equilibrium as stable or unstable}.
    \vspace*{\stretch{2}}
    \part[2]
    On the graph, \textbf{draw the path that investment takes} (i.e. the dynamics of the investment) if average investment begane at $A$ and show where invetsment would eventually end up.
\end{parts}

\newpage
%%%%%%%%%%%%%%%%%%%%%%%%%%%%%%%%%%%%%%%%%%%%%%%%%%%%
%  QUESTION 12
%%%%%%%%%%%%%%%%%%%%%%%%%%%%%%%%%%%%%%%%%%%%%%%%%%%%
\question
In the \textbf{O-Ring Model}, a firm's output is given by $BF(q_{i},q_{j}) = q_{i}q_{j}$ where $0 \leq q_{i},q_{j} \leq 1$ and $B > 0$ is some positive constant.

Let $B = 20$ and $q_{i} = 0.8$ and $q_{j} = 0.5$.

\begin{parts}
    \part[3]
    A firm has two choices: Hire a \textbf{2-person team} of $q_{i}$ workers or hire a \textbf{4-person team} of $q_{j}$ workers.
    Compute the output for both possibilities.
    \vspace*{\stretch{1}}
    \part[3]
    If the firm can add another worker to the \textbf{2-person team} and their $q = 0.7$, what happens to output?
    Compute it
    \vspace*{\stretch{1}}
    \part[4]
    Explain intuitively why increasing the team size is not necessarily a good outcome for firms
    \vspace*{\stretch{1}}
\end{parts}

%-------------------------------------------------------------------------------
\newpage
\section*{Multi-Part Analysis - 40 Points}
Answer the following questions to the best of your ability. 
For full credit, show all of your work and clearly indicate your final solution for each part by circling the answer.
%-------------------------------------------------------------------------------
%%%%%%%%%%%%%%%%%%%%%%%%%%%%%%%%%%%%%%%%%%%%%%%%%%%%
%  QUESTION 13
%%%%%%%%%%%%%%%%%%%%%%%%%%%%%%%%%%%%%%%%%%%%%%%%%%%%
\question
You are hired by the government of Stars Hollow to analyze economic growth in the country. 
Taylor Doose, the government official that hired you, wants you to use the \textbf{Solow Model} of growth to predict what will occur in the economy under the following scenarios.

\begin{parts}
    \part[5]
    What will happen to \textbf{capital per worker $(k)$} and \textbf{output per worker $(y)$} in population growth in Stars Hollow decreases?
    Show graphically.
    \vspace*{\stretch{1}}
    \part[5]
    What will happen to to \textbf{capital per worker $(k)$} and \textbf{output per worker $(y)$} if workers save less of their income? 
    Show graphically.
    \vspace*{\stretch{1}}
    \newpage
    \part[15]
    Starts Hollow is particularly concerned about one situation:
    What will happen to \textbf{capital per worker $(k)$} if \emph{both} population growth decreases \emph{AND} the savings rate decreases?
    Show graphically and explain your answer.

    \textbf{Hint: There may multiple cases to analyze - be sure to discuss each one.}
    \newpage
    \part[5]
    Stars Hollow always competes with Hartford, a neighboring country.
    The Solow model predicts that countries with the \textbf{same savings rate and population growth} should converge to the same \textbf{income per worker $(y)$}.
    Why would it be the case that \textbf{convergence between} Stars Hollow and Hartford may not occur?
    \vspace*{\stretch{1}}
    \part[10]
    Models of economic growth that emphasize savings have been shown to consistently overestimate economic growth in developing countries.
    \textbf{Identify the key assumption(s) of savings.}
    Explain how the failure of the critical assumption(s) in these models result in an overestimate of economic growth.
    \vspace*{\stretch{1}}
\end{parts}

\newpage
\thispagestyle{empty}
%%%%%%%%%%%%%%%%%%%%%%%%%%%%%%%%%%%%%%%%%%%%%%%%%%%%
%  SCRATCH PAPER
%%%%%%%%%%%%%%%%%%%%%%%%%%%%%%%%%%%%%%%%%%%%%%%%%%%%
\subsection*{Scratch Paper}

\newpage
\thispagestyle{empty}
%%%%%%%%%%%%%%%%%%%%%%%%%%%%%%%%%%%%%%%%%%%%%%%%%%%%
%  SCRATCH PAPER
%%%%%%%%%%%%%%%%%%%%%%%%%%%%%%%%%%%%%%%%%%%%%%%%%%%%
\subsection*{Scratch Paper}

\end{questions}

\end{document}