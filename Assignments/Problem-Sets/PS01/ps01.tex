\documentclass[12pt]{exam}

% ---------------------------------------------------
% Page Layout
% ---------------------------------------------------
\usepackage{geometry}   % Flexible page dimensions
\geometry{margin=1in, centering}

% ---------------------------------------------------
% Encoding and Fonts
% ---------------------------------------------------
\usepackage[utf8]{inputenc}   % Input encoding
\usepackage[T1]{fontenc}      % Output font encoding
\usepackage{lmodern}          % Latin Modern fonts (improves output)
\usepackage[official]{eurosym}

% ---------------------------------------------------
% Tables and Columns
% ---------------------------------------------------
\usepackage{multicol}   % Multiple column environments
\usepackage{multirow}   % Multi-row cells in tables
\usepackage{booktabs}   % Better looking tables

% ---------------------------------------------------
% Graphics and Figures
% ---------------------------------------------------
\usepackage{graphicx}   % Include graphics
\usepackage{caption}    % Customization of captions

% ---------------------------------------------------
% Text Styling
% ---------------------------------------------------
\usepackage{color}      % Color support
\usepackage{soul}       % Highlighting, underlining, strikethrough, etc.

% ---------------------------------------------------
% Headers, Footers, and Symbols
% ---------------------------------------------------
\usepackage{lastpage}   % Reference last page in document
\usepackage{bbding}     % Special symbols (checkmarks, etc.)
\usepackage{pmboxdraw}  % Box drawing characters
\usepackage{fontawesome} % Allows customization of icons 

% ---------------------------------------------------
% Section Formatting
% ---------------------------------------------------
\usepackage{titlesec}   % Customize section titles

% ---------------------------------------------------
% Hyperlinks and URLs
% ---------------------------------------------------
\usepackage{url}        % Simple URL typesetting
\usepackage{hyperref}   % Clickable hyperlinks


\titleformat{\section}
  {\normalfont\large\bfseries\sffamily}{\thesection}{1em}{}

\titleformat{\subsection}
  {\normalfont\normalsize\bfseries\sffamily}{\thesection}{1em}{}
  
\hypersetup{
    colorlinks=true,
    linkcolor=blue,
    filecolor=magenta,      
    urlcolor=blue,
    citecolor=blue,
}

% ---------------------------------------------------
% Multiple Choice Options 
% ---------------------------------------------------
\CorrectChoiceEmphasis{\color{red}\bfseries\itshape}

% ---------------------------------------------------
% Defines Header & Footer Content
% ---------------------------------------------------
% This is the name of the course (left header)
\newcommand{\class}{\fontfamily{lmss} \textbf{EC 390}} 
% This is the name of the assignment (center header)
\newcommand{\examnum}{\fontfamily{lmss} \textbf{PS 01}} 
% This is the due date (right header)
\newcommand{\examdate}{\fontfamily{lmss} \textbf{October 08 at 11:59am}} 

% ---------------------------------------------------
% Points Options
% ---------------------------------------------------
\addpoints              % Allows to add points up to include table at end of document
\bracketedpoints        % Puts points per question inside brackets instead of parenthesis 

% ---------------------------------------------------
% Header & Footer Options
% ---------------------------------------------------
\pagestyle{headandfoot}         % Fancy equivalent for exam documentclass
\firstpageheadrule              % Horizontal bar in first page
\runningheadrule                % Horizontal bar in rest of pages

% 1st page header content
\firstpageheader{\class}{}{\fontfamily{lmss} Due \examdate} 
% Rest of pages header content
\runningheader{\class}{\examnum}{\fontfamily{lmss} Due \examdate} 

% 1st page footer content
\firstpagefooter{\fontfamily{lmss} Points earned: \makebox[1in]{\hrulefill} / \pointsonpage{\thepage} points}{}{\thepage\ of \numpages} 
% Rest of pages footer content
\runningfooter{\fontfamily{lmss} Points earned: \makebox[1in]{\hrulefill} / \pointsonpage{\thepage} points}{}{\thepage\ of \numpages} 

% ---------------------------------------------------
% Section Formatting & Options
% ---------------------------------------------------
\titleformat{\section}
  {\normalfont\large\bfseries\sffamily}{\thesection}{1em}{}

\titleformat{\subsection}
  {\normalfont\normalsize\bfseries\sffamily}{\thesection}{1em}{}

% ---------------------------------------------------
% Body Begins
% ---------------------------------------------------
%\printanswers           % Comment out to hide answers 

\begin{document}

\fontfamily{lmss}\selectfont

\begin{center}
    \textbf{{\LARGE EC 390 Problem Set 01}} \\
    \bigskip 
\end{center}

% ---------------------------------------------------
% Instructions
% ---------------------------------------------------
\noindent \textbf{Instructions:} 
Answers must be submitted online through the designated Canvas assignment in a \textbf{PDF file}.
Any other file type is not allowed. 
This Problem Set is due on \examdate.
Please write as legible and clearly as possible. 
You will not be given full credit if your answers cannot be easily understood. 

% ---------------------------------------------------
% Questions Begins
% ---------------------------------------------------
\section*{Questions}

\begin{questions}
    
\question[7] 
What does it mean if someone is living in \emph{absolute poverty}? 
What is the World Bank's definition of absolute povery?
Does it make sense to define poverty in terms of income? 
Why or why not? 
(No more than 1 or 2 paragraphs)

\begin{solution}
    \vspace*{2in}
\end{solution}

\vspace*{\stretch{1}}       % Standardizes distance between questions in page

\question[3] 
The \textbf{Purchasing Power Parity (PPP)} is defined to be (in your own words): 

\vspace*{\stretch{1}} 

\question[10]
Use the following information to determine whether there is evidence of \textbf{relative convergence, absolute convergence, or both}:

\begin{table}[ht]
  \centering
  \label{tab:convergence}
  \begin{tabular}{lcc}
    \toprule
    Country & GNI per Capita (2010) & GNI per Capita (2019) \\
    \midrule
    Megaton & \$545 & \$890.21 \\
    New Vegas & \$350 & \$458.61 \\
    \bottomrule
  \end{tabular}
\end{table}
\vspace*{\stretch{1}}

\newpage

\question
The prices of goods in Maynooth and Eugene are given below. 
Note that the prices from Maynooth are measured in \euro $\,$ and prices in Eugene are measured in \$.

\begin{table}[ht]
  \centering
  \label{tab:prices}
  \begin{tabular}{lcc}
    \toprule
     & Maynooth & Eugene \\
    \midrule
    Shoes & \euro 45 & \$ 55 \\
    Sweaters & \euro 35 & \$ 20 \\
    Hat & \euro 20 & \$ 15 \\
    \bottomrule
  \end{tabular}
\end{table}

\begin{parts}
  \part[4] 
  Suppose that we define a basket of goods as \textbf{two pairs of shoes} and \textbf{one sweater}.
  What is the \textbf{purchasing power parity} between Maynooth and Eugene $(P_{May, Eug})$?
  \vspace*{\stretch{1}}
  \part[3] 
  Suppose that the GNI per capita of Maynooth is \euro $\,$ 1,500. 
  Using part (a), what is the \textbf{PPP-adjusted} value of Maynooth's GNI per capita (in \$)?
  \vspace*{\stretch{1}}
  \part[3]
  Suppose that the GNI per capita of Eugene is \$ 2,500.
  Using part (a), what is the \textbf{PPP-adjusted} value of Eugene's GNI per capita (in \euro)? 
  \vspace*{\stretch{1}}
\end{parts}

\newpage 

\question 
Consider the city of Hamilton, New York. 
Let Hamilton follow the Harrod-Domar model of growth.

\begin{parts}
  \part[3] 
  Hamilton's ratio of capital-output is 5.5 and has a savings rate of 22\%.
  At what rate do you expect Hamilton's output to grow?
  \vspace*{\stretch{1}}
  \part[3] 
  Hamilton wants to increase the city's growth rate to 7.7\%.
  If Hamilton's capital-output ratio remains at 5.5, what savings rate should Hamilton target if it wants to achieve this growth rate?
  \vspace*{\stretch{1}}
  \part[4]
  \textbf{True or False?}
  Developing countris can change their savings rate easily. 
  Explain. 
  (Be brief, no more than 1 paragraph)
  \vspace*{\stretch{1}}
\end{parts}

\newpage

\question[10]
Consider the Solow model discussed in class.
Consider a population growth rate increase from $n$ to $n'$.
Show graphically what happens to \textbf{capital per worker $(k)$} and \textbf{output per worker $(y)$}.

\end{questions}

\end{document}