\documentclass[12pt]{exam}

% ---------------------------------------------------
% Page Layout
% ---------------------------------------------------
\usepackage{geometry}   % Flexible page dimensions
\geometry{margin=1in, centering}

% ---------------------------------------------------
% Encoding and Fonts
% ---------------------------------------------------
\usepackage[utf8]{inputenc}   % Input encoding
\usepackage[T1]{fontenc}      % Output font encoding
\usepackage{lmodern}          % Latin Modern fonts (improves output)
\usepackage{amsmath}          % Math package

% ---------------------------------------------------
% Tables and Columns
% ---------------------------------------------------
\usepackage{multicol}   % Multiple column environments
\usepackage{multirow}   % Multi-row cells in tables
\usepackage{booktabs}   % Better looking tables
\usepackage{paralist}   % Horizontal lists

% ---------------------------------------------------
% Graphics and Figures
% ---------------------------------------------------
\usepackage{graphicx}   % Include graphics
\usepackage{caption}    % Customization of captions

% ---------------------------------------------------
% Text Styling
% ---------------------------------------------------
\usepackage{color}      % Color support
\usepackage{soul}       % Highlighting, underlining, strikethrough, etc.

% ---------------------------------------------------
% Headers, Footers, and Symbols
% ---------------------------------------------------
\usepackage{lastpage}   % Reference last page in document
\usepackage{bbding}     % Special symbols (checkmarks, etc.)
\usepackage{pmboxdraw}  % Box drawing characters
\usepackage{fontawesome} % Allows customization of icons 

% ---------------------------------------------------
% Section Formatting
% ---------------------------------------------------
\usepackage{titlesec}   % Customize section titles

% ---------------------------------------------------
% Hyperlinks and URLs
% ---------------------------------------------------
\usepackage{url}        % Simple URL typesetting
\usepackage{hyperref}   % Clickable hyperlinks


\titleformat{\section}
  {\normalfont\large\bfseries\sffamily}{\thesection}{1em}{}

\titleformat{\subsection}
  {\normalfont\normalsize\bfseries\sffamily}{\thesection}{1em}{}
  
\hypersetup{
    colorlinks=true,
    linkcolor=blue,
    filecolor=magenta,      
    urlcolor=blue,
    citecolor=blue,
}

% ---------------------------------------------------
% Multiple Choice Options 
% ---------------------------------------------------
\CorrectChoiceEmphasis{\color{red}\bfseries\itshape}

% ---------------------------------------------------
% Defines Header & Footer Content
% ---------------------------------------------------
% This is the name of the course (left header)
\newcommand{\class}{\fontfamily{lmss} \textbf{EC 390}} 
% This is the name of the assignment (center header)
\newcommand{\examnum}{\fontfamily{lmss} \textbf{PS 04}} 
% This is the due date (right header)
\newcommand{\examdate}{\fontfamily{lmss} \textbf{November 27 at 11:59pm}} 

% ---------------------------------------------------
% Points Options
% ---------------------------------------------------
\addpoints              % Allows to add points up to include table at end of document
\bracketedpoints        % Puts points per question inside brackets instead of parenthesis 
%\printanswers
\unframedsolutions

% ---------------------------------------------------
% Header & Footer Options
% ---------------------------------------------------
\pagestyle{headandfoot}         % Fancy equivalent for exam documentclass
\firstpageheadrule              % Horizontal bar in first page
\runningheadrule                % Horizontal bar in rest of pages

% 1st page header content
\firstpageheader{\class}{}{\fontfamily{lmss} Due \examdate} 
% Rest of pages header content
\runningheader{\class}{\examnum}{\fontfamily{lmss} Due \examdate} 

% 1st page footer content
\firstpagefooter{\fontfamily{lmss} Points earned: \makebox[1in]{\hrulefill} / \pointsonpage{\thepage} points}{}{\thepage\ of \numpages} 
% Rest of pages footer content
\runningfooter{\fontfamily{lmss} Points earned: \makebox[1in]{\hrulefill} / \pointsonpage{\thepage} points}{}{\thepage\ of \numpages} 

% ---------------------------------------------------
% Section Formatting & Options
% ---------------------------------------------------
\titleformat{\section}
  {\normalfont\large\bfseries\sffamily}{\thesection}{1em}{}

\titleformat{\subsection}
  {\normalfont\normalsize\bfseries\sffamily}{\thesection}{1em}{}

% ---------------------------------------------------
% Body Begins
% ---------------------------------------------------
\begin{document}

\fontfamily{lmss}\selectfont

\begin{center}
    \textbf{{\LARGE EC 390 Problem Set 04}} \\
    \bigskip 
\end{center}

% ---------------------------------------------------
% Instructions
% ---------------------------------------------------
\noindent \textbf{Instructions:} 
Answers must be submitted online through the designated Canvas assignment in a \textbf{PDF file}.
Any other file type is not allowed. 
This Problem Set is due on \examdate.
Please write as legible and clearly as possible. 
You will not be given full credit if your answers cannot be easily understood. 

% ---------------------------------------------------
% Questions Begins
% ---------------------------------------------------
\section*{Questions}

\begin{questions}
    
% ===================================================
% QUESTION 01
% ===================================================
\question 
Use the graph below to answer the following questions:

\begin{figure}[ht!]
  \centering
  \includegraphics[width=\textwidth]{images/todaro-question.png}
\end{figure}

\begin{parts}
    \part[3]
    Suppose $w^{C} = 18$. How many people work in the city?
    \begin{solution}
      1
    \end{solution}
    \vspace*{\stretch{1}} 
    \part[10]
    Suppose $w^{C} = 18$. Solve for the equilibrium and show it graphically.
    Show \textbf{ALL} your work to get the equilibrium
    \begin{solution}
      \begin{align*}
        E(w^{C}) = w^{R} = \dfrac{1}{10 - L^{R}} * 18
      \end{align*}
      \begin{center}
      \begin{tabular}{l|ccc}
        $L^{R}$ & $E(w^{C})$ & $w^{R}$ & Equilibrium? \\
        \hline
        9       &     18     &    1    &      No      \\ 
        7       &      6     &    2    &      No      \\ 
        4       &      3     &    3    &     Yes      \\ 
      \end{tabular}
    \end{center}
    \end{solution}
    \vspace*{\stretch{3}} 
\end{parts}

\newpage
% ===================================================
% QUESTION 02
% ===================================================
\question 
Using the demand and supply functions below, answer the following questions:

$$
  \textbf{Demand:} \;\; P = 120 - \dfrac{4}{7}Q_{d} \;\;\; ; \;\;\; \textbf{Supply:} \;\; P = \dfrac{1}{4}Q_{s}
$$

Suppose a \textbf{Developing Nation} opens up to \textbf{free-trade} and now faces a world price, $P_{w} = 25$.

\begin{parts}
  \part[5] 
  Sketch the market with the \textbf{new price line} and corresponding equilibria points for \textbf{quantity demanded and supplied}
  \begin{solution}
      \begin{center}
        \includegraphics[width=0.75\textwidth]{images/trade-question.png}
      \end{center}
  \end{solution}
  \vspace*{\stretch{1}} 
  \part[10] 
  Calculate the \textbf{equilibrium values for quantity demanded, quantity supplied, imports, and surplus values}.
  \begin{solution}
    Find $Q_{D}, Q_{S}$ at $P^{w} = 25$ and then find CS and PS

    \begin{multicols}{3}
      \centering
      \textbf{Demand}

      \begin{align*}
        25 &= 120 - \dfrac{4}{7}Q_{d} \\
        \dfrac{4}{7}Q_{d} &= 120 - 25 \\
        Q_{d} &= \dfrac{95 \times 7}{4} = 166.25  
      \end{align*}
      
      \columnbreak

      \textbf{Supply}

      \begin{align*}
        25 &= \dfrac{1}{4}Q_{S} \\ 
        Q_{S} &= 100
      \end{align*}
      
      \columnbreak

      \textbf{Surplus}

      \begin{align*}
        CS &= \dfrac{1}{2}(120 - 25) * 166.25 \\
        CS &= 7,896.88 \\
        \vspace*{0.5cm} \\
        PS &= \dfrac{1}{2}(25) * 100 \\
        PS &= 1250
      \end{align*}
      
      \columnbreak

    \end{multicols}
  \end{solution}
  \vspace*{\stretch{1}}
  \part[2] 
  Should we expect social welfare to be larger or smaller, relative to \textbf{no trade}?
  \begin{solution}
    In theory, it should be larger. In reality, it depends on magnitudes.
  \end{solution}
  \vspace*{\stretch{1}} 

  \newpage
  \part[5] 
  Now suppose \textbf{the government intervenes, setting a tariff rate of} $t = 4$.
  Sketch the updated demand and supply curves.
  Label it properly and \textbf{highlight which regions are the deadweight loss areas}.
  \begin{solution}
    \begin{center} 
      \includegraphics[width=\textwidth]{images/trade-question-02.png}
    \end{center}
  \end{solution}
  \vspace*{\stretch{1}} 
  \part[15] 
  Calculate the equilibria:
  
  \textbf{(1) Quantity Supplied} \hspace{1em} \textbf{(2) Quantity Demanded} \hspace{1em} \textbf{(3) Quantity Imported} \hspace{1em} \textbf{(4) Consumer,Producer Surplus, and Government Revenues}

  \begin{solution}
    \begin{multicols}{3}
      \centering
      \textbf{Supply}
      \begin{align*}
        P^{t} &= \dfrac{1}{4}Q_{S} \\
        29 &= \dfrac{1}{4}Q_{S} \\
        Q_{S} &= 116  
      \end{align*}

      \textbf{Demand}
      \begin{align*}
        29 &= 120 - \dfrac{4}{7} Q_{D} \\
        \dfrac{4}{7} Q_{D} &= 91 \\
        Q_{D} &= 159.25
      \end{align*}

      \columnbreak

      \textbf{Imports}
      \vspace*{-3cm}
      \begin{align*}
        &Q_{D} - Q_{S} \\
        &= 159.25 - 116 \\
        &= 43.25
      \end{align*}

      \columnbreak

      \textbf{Surplus and Gov't Revenue}
      \begin{align*}
        CS &= \dfrac{1}{2}(120 - 29) \cdot 159.25 \\
        CS &= 7,245.88 \\
        PS &= \dfrac{1}{2}(29) \cdot 116 \\
        PS &= 1,682 \\
        \text{Gov't Rev} &= t \cdot imports \\
        &= 4 \cdot 43.25 \\
        &= 173
      \end{align*}
    \end{multicols}
  \end{solution}
  \vspace*{\stretch{1}} 
\end{parts}

\end{questions}
\end{document}