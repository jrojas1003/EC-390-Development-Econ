\documentclass[12pt]{exam}

% ---------------------------------------------------
% Page Layout
% ---------------------------------------------------
\usepackage{geometry}   % Flexible page dimensions
\geometry{margin=1in, centering}

% ---------------------------------------------------
% Encoding and Fonts
% ---------------------------------------------------
\usepackage[utf8]{inputenc}   % Input encoding
\usepackage[T1]{fontenc}      % Output font encoding
\usepackage{lmodern}          % Latin Modern fonts (improves output)
\usepackage{amsmath}          % Math package
% ---------------------------------------------------
% Tables and Columns
% ---------------------------------------------------
\usepackage{multicol}   % Multiple column environments
\usepackage{multirow}   % Multi-row cells in tables
\usepackage{booktabs}   % Better looking tables

% ---------------------------------------------------
% Graphics and Figures
% ---------------------------------------------------
\usepackage{graphicx}   % Include graphics
\usepackage{caption}    % Customization of captions
\usepackage{subcaption}

% ---------------------------------------------------
% Text Styling
% ---------------------------------------------------
\usepackage{color}      % Color support
\usepackage{soul}       % Highlighting, underlining, strikethrough, etc.

% ---------------------------------------------------
% Headers, Footers, and Symbols
% ---------------------------------------------------
\usepackage{lastpage}   % Reference last page in document
\usepackage{bbding}     % Special symbols (checkmarks, etc.)
\usepackage{pmboxdraw}  % Box drawing characters
\usepackage{fontawesome} % Allows customization of icons 

% ---------------------------------------------------
% Section Formatting
% ---------------------------------------------------
\usepackage{titlesec}   % Customize section titles

% ---------------------------------------------------
% Hyperlinks and URLs
% ---------------------------------------------------
\usepackage{url}        % Simple URL typesetting
\usepackage{hyperref}   % Clickable hyperlinks


\titleformat{\section}
  {\normalfont\large\bfseries\sffamily}{\thesection}{1em}{}

\titleformat{\subsection}
  {\normalfont\normalsize\bfseries\sffamily}{\thesection}{1em}{}
  
\hypersetup{
    colorlinks=true,
    linkcolor=blue,
    filecolor=magenta,      
    urlcolor=blue,
    citecolor=blue,
}

% ---------------------------------------------------
% Multiple Choice Options 
% ---------------------------------------------------
\CorrectChoiceEmphasis{\color{red}\bfseries\itshape}

% ---------------------------------------------------
% Defines Header & Footer Content
% ---------------------------------------------------
% This is the name of the course (left header)
\newcommand{\class}{\fontfamily{lmss} \textbf{EC 390}} 
% This is the name of the assignment (center header)
\newcommand{\examnum}{\fontfamily{lmss} \textbf{PS 03}} 
% This is the due date (right header)
\newcommand{\examdate}{\fontfamily{lmss} \textbf{November 13 at 01:59pm}} 

% ---------------------------------------------------
% Points Options
% ---------------------------------------------------
\addpoints              % Allows to add points up to include table at end of document
\bracketedpoints        % Puts points per question inside brackets instead of parenthesis 
%\printanswers
% ---------------------------------------------------
% Header & Footer Options
% ---------------------------------------------------
\pagestyle{headandfoot}         % Fancy equivalent for exam documentclass
\firstpageheadrule              % Horizontal bar in first page
\runningheadrule                % Horizontal bar in rest of pages

% 1st page header content
\firstpageheader{\class}{}{\fontfamily{lmss} Due \examdate} 
% Rest of pages header content
\runningheader{\class}{\examnum}{\fontfamily{lmss} Due \examdate} 

% 1st page footer content
\firstpagefooter{\fontfamily{lmss} Points earned: \makebox[1in]{\hrulefill} / \pointsonpage{\thepage} points}{}{\thepage\ of \numpages} 
% Rest of pages footer content
\runningfooter{\fontfamily{lmss} Points earned: \makebox[1in]{\hrulefill} / \pointsonpage{\thepage} points}{}{\thepage\ of \numpages} 

% ---------------------------------------------------
% Section Formatting & Options
% ---------------------------------------------------
\titleformat{\section}
  {\normalfont\large\bfseries\sffamily}{\thesection}{1em}{}

\titleformat{\subsection}
  {\normalfont\normalsize\bfseries\sffamily}{\thesection}{1em}{}

% ---------------------------------------------------
% Body Begins
% ---------------------------------------------------
\begin{document}

\fontfamily{lmss}\selectfont

\begin{center}
    \textbf{{\LARGE EC 390 Problem Set 03}} \\
    \bigskip 
\end{center}

% ---------------------------------------------------
% Instructions
% ---------------------------------------------------
\noindent \textbf{Instructions:} 
Answers must be submitted online through the designated Canvas assignment in a \textbf{PDF file}.
Any other file type is not allowed. 
This Problem Set is due on \examdate.
Please write as legible and clearly as possible. 
You will not be given full credit if your answers cannot be easily understood. 

% ---------------------------------------------------
% Questions Begins
% ---------------------------------------------------
\section*{Questions}

\begin{questions}
  
% Question 01
\question 
Suppose that 2 farmers enter into an agreement where one farmer grows \textbf{carrots} and the other grows \textbf{leeks}. 
The table below shows their yields over the past 5 years.

\begin{table}[h!]
    \centering
\begin{tabular}{l|ccccc}
    Crop    & Year 1 & Year 2 & Year 3 & Year 4 & Year 5 \\
    \hline 
    Carrots &    5    &   3    &   8    &   4    &   10   \\
    Leeks   &    12   &   17   &   0    &   14   &    3   \\    
\end{tabular} 
\end{table}

\begin{parts}
    \part[5] 
    Solve for the \textbf{expected yield} of \textbf{carrots} and \textbf{leeks} in year 6. 
    Show your work.
    \begin{solution}
      \begin{align*}
        \text{Expected Yield of Carrots:} \;\; \dfrac{5+3+8+4+10}{5} = 6\\ \\
        \text{Expected Yield of Leeks:} \;\; \dfrac{12+17+0+14+3}{5} = 9.2
      \end{align*}
    \end{solution} 
    \vspace*{\stretch{1}}
    \part[4]
    Do \textbf{carrots} and \textbf{leeks} appear to have a positive or negative covariance?
    Does this mean that these veggies are a good or bad choice for diversification?
    (You do not need to calculate the covariance to answer this question)
    \begin{solution}
      They have a negative covariance. 
      Years in which carrots have a high yield, leeks yield is low, and vice-versa. 

      This means that they are a good choice for diversification, as they serve as informal insurance for each other. 
    \end{solution}
    \vspace*{\stretch{1}}
\end{parts}

\newpage
% Question 02
\question
Recall a \textbf{sharecropping} agreement is where a farmer provides a percentage of their yield to a landowner in exchange for renting the land.
Consider the graph below and answer the following questions (include calculations in your work).

\begin{figure}[ht!]
  \centering
  \includegraphics[width=0.5\textwidth, height=0.4\textheight]{images/sharecropping.PNG}
\end{figure}

\begin{parts}
  \part[5] 
  Suppose the farmer has to pay 0\% of their yield to the landowner, and the farmer's next best alternative is working informally for \$5 an hour. 
  Find the \textbf{optimal amount of labor} for the farmer. 
  \begin{solution}
    With an outside option of \$5 an hour, the optimal choice of labor by the farmer would be 30 hours when they get to keep all of their yield.
  \end{solution}
  \vspace*{\stretch{1}}
  \part[5] 
  Now consider the case where the farmer keeps only $\alpha$\% (as shown in the graph).
  Calculate the \textbf{farmer's profit, landlord's profit, and the deadweight loss} associated with this sharecropping agreement. 
  (Answers should be numeric)
  \begin{solution}
    The area of $A$ is the deadweight loss, the area of $B$ is the farmer's profit, and the area of $C$ is the landlord's profit.
    \begin{align*}
      A &= 1/2 * 10 * 2 = 10 \\
      B &= 1/2 * 20 * 3 = 30 \\
      C &= (1/2 * 30 * 6) - A - B = 50
    \end{align*}
  \end{solution}
  \vspace*{\stretch{1}}
  \part[3] 
  What is the benefit to the farmer from sharecropping?
  \begin{solution}
    Sharecropping is beneficial to the farmer because it serves as a form of insurance against a bad crop. 
    It helps the farmer spread the risk at the cost of sharing the yields.
  \end{solution}
  \vspace*{\stretch{1}}
\end{parts}

\newpage
% Question 03
\question[2]
What is \textbf{urban bias}? 
Why can this lead to negative outcomes for society?
\begin{solution}
  \textbf{Urban bias} is the favoring of urban areas over rural ones. 
  For example, greater resources are invested in cities than in rural zones.

  This can lead to negative outcomes because as rural areas are left behind, inequalities can grow larger and possibly lead to political unrest.
\end{solution}
\vspace*{\stretch{1}}
% Question 04
\question 
Use the following graph on population age for the Philippines and Japan to answer the following questions:

\begin{figure}[h!]
    \centering
    \begin{subfigure}[]{0.45\textwidth}
        \includegraphics[width=0.9\textwidth]{images/japanage.png}
        \label{fig:first}
    \end{subfigure}
    \begin{subfigure}[]{0.45\textwidth}
        \includegraphics[width=0.9\textwidth]{images/philippinesage.png}
    \end{subfigure}
    \label{fig:sidebyside}
\end{figure}

\begin{parts}
  \part[3] 
  On average, which country has a younger population?
  \begin{solution}
    Philippines
  \end{solution}
  \vspace*{\stretch{1}}
  \part[3] 
  How might the age structure of the Philippines influence the \textbf{dependency burden} of the Philippines?
  \begin{solution}
    It would shift the dependency burden toward young children (under the working age)
  \end{solution}
  \vspace*{\stretch{1}}
  \part[3] 
  How might the age structure of Japan influence the \textbf{dependency burden} of Japan?
  \begin{solution}
    It would shift the dependency burden toward the older population (retirees)
  \end{solution}
  \vspace*{\stretch{1}}
\end{parts}

\newpage
% Question 05
\question 
Use the following graph on \textbf{household preferences} to answer the following questions and show your work for each answer

\begin{parts}
  \part[4] 
  Suppose the household budget is \$100.
  If the household \textbf{spends all of their budget on children}, they can afford to have 8 children. 
  If they \textbf{spend all of their budget on food}, they can afford 40 units.
  What is the \textbf{price of a child and the price of food}?
  \begin{solution}
    \begin{align*}
      \text{Price of a child:} \;\; P_{c} = \dfrac{Budget}{Units} = \dfrac{100}{8} = 12.5 \\ \\
      \text{Price of food:} \;\; P_{f} = \dfrac{Budget}{Units} = \dfrac{100}{40} = 2.5
    \end{align*}
  \end{solution}
  \vspace*{\stretch{1}}
  \part[4] 
  \textbf{On the graph below}, draw the \textbf{households budget line} and determine the \textbf{optimal number of children} and the \textbf{optimal amount of food} the household will consume.
  \begin{figure}[ht!]
  \centering
  \includegraphics[width=0.45\textheight]{images/fertility1.PNG}
  \end{figure}
  \begin{solution}
    $Q^{*}_{C} = 6 \;\;, \;\; Q^{*}_{f} = 10$
  \end{solution}
  \vspace*{\stretch{1}}
  \newpage
  \part[5] 
  Suppose the government is concerned that the fertility rate in the society is too high. 
  To remedy this, the government imposes a ``four-child policy''. 
  This policy mandates that each household can have at most four children. 
  Under this policy, what is the households \textbf{new optimal point of consumption of children and food}?
  \begin{figure}[ht!]
  \centering
  \includegraphics[width=0.5\textheight]{images/fertility1.PNG}
  \end{figure}
  \begin{solution}
    $Q^{*}_{C} = 4 \;\;, \;\; Q^{*}_{f} = 20$
  \end{solution}
  \vspace*{\stretch{0.5}}
  \part[4] 
  Under the ``four-child policy'' is the \textbf{household better off, worse off, or neither}?
  How can you tell?
  \begin{solution}
    The household is \textbf{worse off}. 
    You can tell this because they are at a \textbf{lower indifference curve} due to the limit in children. 
    Any time we have to pick a lower (toward the origin) indifference curve we know they are worse off. 
  \end{solution}
  \vspace*{\stretch{1}}
\end{parts} 

\end{questions}

\end{document}