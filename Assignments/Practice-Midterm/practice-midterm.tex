\documentclass[12pt]{exam}

% ---------------------------------------------------
% Page Layout
% ---------------------------------------------------
\usepackage{geometry}   % Flexible page dimensions
\geometry{margin=1in, centering}

% ---------------------------------------------------
% Encoding and Fonts
% ---------------------------------------------------
\usepackage[utf8]{inputenc}   % Input encoding
\usepackage[T1]{fontenc}      % Output font encoding
\usepackage{lmodern}          % Latin Modern fonts (improves output)

% ---------------------------------------------------
% Tables and Columns
% ---------------------------------------------------
\usepackage{multicol}   % Multiple column environments
\usepackage{multirow}   % Multi-row cells in tables
\usepackage{booktabs}   % Better looking tables

% ---------------------------------------------------
% Graphics and Figures
% ---------------------------------------------------
\usepackage{graphicx}   % Include graphics
\usepackage{caption}    % Customization of captions

% ---------------------------------------------------
% Text Styling
% ---------------------------------------------------
\usepackage{color}      % Color support
\usepackage{soul}       % Highlighting, underlining, strikethrough, etc.

% ---------------------------------------------------
% Headers, Footers, and Symbols
% ---------------------------------------------------
\usepackage{lastpage}   % Reference last page in document
\usepackage{bbding}     % Special symbols (checkmarks, etc.)
\usepackage{pmboxdraw}  % Box drawing characters
\usepackage{fontawesome} % Allows customization of icons 

% ---------------------------------------------------
% Section Formatting
% ---------------------------------------------------
\usepackage{titlesec}   % Customize section titles

% ---------------------------------------------------
% Hyperlinks and URLs
% ---------------------------------------------------
\usepackage{url}        % Simple URL typesetting
\usepackage{hyperref}   % Clickable hyperlinks


\titleformat{\section}
  {\normalfont\large\bfseries\sffamily}{\thesection}{1em}{}

\titleformat{\subsection}
  {\normalfont\normalsize\bfseries\sffamily}{\thesection}{1em}{}
  
\hypersetup{
    colorlinks=true,
    linkcolor=blue,
    filecolor=magenta,      
    urlcolor=blue,
    citecolor=blue,
}

% ---------------------------------------------------
% Multiple Choice Options 
% ---------------------------------------------------
\CorrectChoiceEmphasis{\color{red}\bfseries\itshape}
%\printanswers
% ---------------------------------------------------
% Defines Header & Footer Content
% ---------------------------------------------------
% This is the name of the course (left header)
\newcommand{\class}{\fontfamily{lmss} \textbf{EC 390}} 
% This is the name of the assignment (center header)
\newcommand{\examnum}{\fontfamily{lmss} \textbf{Practice Midterm}} 
% This is the due date (right header)
\newcommand{\examdate}{\fontfamily{lmss} \textbf{}} 

% ---------------------------------------------------
% Points Options
% ---------------------------------------------------
\addpoints              % Allows to add points up to include table at end of document
\bracketedpoints        % Puts points per question inside brackets instead of parenthesis 

% ---------------------------------------------------
% Header & Footer Options
% ---------------------------------------------------
\pagestyle{headandfoot}         % Fancy equivalent for exam documentclass
\firstpageheadrule              % Horizontal bar in first page
\runningheadrule                % Horizontal bar in rest of pages

% 1st page header content
\firstpageheader{\class}{\examnum}{\fontfamily{lmss} \examdate} 
% Rest of pages header content
\runningheader{\class}{\examnum}{\fontfamily{lmss} \examdate} 

% 1st page footer content
\firstpagefooter{\fontfamily{lmss} Points earned: \makebox[1in]{\hrulefill} / \pointsonpage{\thepage} points}{}{\thepage\ of \numpages} 
% Rest of pages footer content
\runningfooter{\fontfamily{lmss} Points earned: \makebox[1in]{\hrulefill} / \pointsonpage{\thepage} points}{}{\thepage\ of \numpages} 

% ---------------------------------------------------
% Section Formatting & Options
% ---------------------------------------------------
\titleformat{\section}
  {\normalfont\large\bfseries\sffamily}{\thesection}{1em}{}

\titleformat{\subsection}
  {\normalfont\normalsize\bfseries\sffamily}{\thesection}{1em}{}

\title{Midterm Exam}
\author{EC 390 - Development Economics}
\date{Fall 2025} 

% ---------------------------------------------------
% Body Begins
% ---------------------------------------------------
\begin{document}

\begin{coverpages}

\maketitle

\begin{center}
\fbox{\fbox{\parbox{5.5in}{\centering
Give every question your best attempt. \\
Best of luck. \\
You got this!
}}}\end{center}
\vspace{2cm}
\makebox[0.6\textwidth]{Name:\enspace\hrulefill}
\makebox[0.35\textwidth]{95\#:\enspace\hrulefill}

\vspace{1cm}
\noindent
The maximum amount of points on this exam is 75 points. 
You have a total of 1h 20min (80 minutes) to complete the exam, unless otherwise noted. 
The only items allowed on your desk at any time are a pen and/or pencil, scratch paper, a 3x5 note card, and a calculator. 
Everything else must be stored in your bag underneath your desk. 
Any form of cheating will result on a zero on the exam.\\

\noindent There are three sections to be completed:

\begin{itemize}
    \item \textbf{Multiple Choice:} 5 Questions
    \item \textbf{Short Answer Questions:} 2 Questions
    \item \textbf{Multi-Part Analysis Questions:} 1 Question (6 parts)
\end{itemize}

\noindent Point totals and question specific instructions are listed for each section.
Please ask for clarification if a question is not clear to you.\\

\noindent The exam is a total of \numpages $\,$ pages. 
There are 5 pages of questions + 2 pages for scratch paper.

\noindent \textbf{Please verify you have all \numpages $\,$ in your exam. If you do not, let me know immediately.}

\end{coverpages}

% ---------------------------------------------------
% Multiple Choice
% ---------------------------------------------------
\section*{Multiple Choice - 25 Points}
Circle or "X" the answer you think most correctly answers the following questions. 
If you mark a choice and would like to change it, \textbf{clearly indicate which one is your correct answer}. 

\begin{questions}
\question
Consider the following statistics:
Country A has a GNI of 5,500. 
Income earned by residents abroad is equal to 800.
Income earned by non-residents domestically is 300.

What is the GDP of Country A?

\begin{choices}
    \choice 4,700
    \CorrectChoice 5,000
    \choice 5,300
    \choice 5,800
\end{choices}

\question
In the \textbf{Lewis two-sector model}, the turning point occurs when:

\begin{choices}
    \choice The industrial/modern sector stops reinvesting profits
    \choice The marginal product of labor in agriculture becomes positive
    \CorrectChoice Surplus labor in the subsistence/traditional sector is fully absorbed
    \choice The savings rate in the industrial/modern sector falls below the population growth rate
\end{choices}

\question 
In growth models that explain economic expansion primarily through capital accumulation (such as the Harrod–Domar and Solow models), which key assumption links savings behavior to the growth of output?

\begin{choices}
    \CorrectChoice All savings become investments
    \choice All savings are held in cash
    \choice Investment depends on government spending
    \choice Savings and investments are determined independently
\end{choices}

\question 
Suppose there are two firms in a developing economy. 
In order to become profitable, a \textbf{large initial investment is required}, but it requires both firms to operate simultaneously (ex. a car factory and a steel mill).

Which concept best describes this situation?

\begin{choices}
    \choice Capital deepening
    \choice Comparative advantage
    \choice Crowding out
    \CorrectChoice Coordination Failure
\end{choices}

% ---------------------------------------------------
% Short Answer
% ---------------------------------------------------
\newpage
\section*{Short Answer - X points}
Answer the following questions to the best of your ability.
For full credit, show all of your work and clearly indicate your final solution for each party by circling the answer.

\question 
What is a \textbf{Big Push}? 
Explain intuitively and illustrate the concept by \textbf{drawing an S-curve diagram with the proper dynamics}. 

% ---------------------------------------------------
% Multi-part Analysis
% ---------------------------------------------------
\newpage
\section*{Multi-Part Analysis - X points}
Answer the following questions to the best of your ability.
For full credit, show all of your work and clearly indicate your final solution for each party by circling the answer.

\question 
An economy can evolve under different growth scenarios. 
As a consultant, you are tasked with helping describe possible shifts/movements/growth/pitfalls \textbf{graphically}.

\begin{parts}
    \part 
    Draw a \textbf{Solow diagram} showing the steady-state level of capital per worker. 
    Label the savings curve and capital growth line. 
    Then, show what happens to $k^{*}$ and $y^{*}$ if production falls. 
    \vspace*{\stretch{1}}
    \part 
    Consider the \textbf{Lewis two-stage model diagram}.
    There is a traditional sector and a modern sector. 
    Show how an increase in \textbf{agricultural productivity} affects 
    (1) The wage in the traditional sector
    \vspace*{\stretch{1}}
    \part 
    How would this increase in \textbf{agricultural productivity} affect the \textbf{pace of labor migration to the modern sector}.
    \vspace*{\stretch{1}}
\end{parts}

\end{questions}

\end{document}