\documentclass[11pt]{article}

% ---------------------------------------------------
% Page Layout
% ---------------------------------------------------
\usepackage{geometry}   % Flexible page dimensions
\geometry{margin=1in, centering}

% ---------------------------------------------------
% Encoding and Fonts
% ---------------------------------------------------
\usepackage[utf8]{inputenc}   % Input encoding
\usepackage[T1]{fontenc}      % Output font encoding
\usepackage{lmodern}          % Latin Modern fonts (improves output)

% ---------------------------------------------------
% Tables and Columns
% ---------------------------------------------------
\usepackage{multicol}   % Multiple column environments
\usepackage{multirow}   % Multi-row cells in tables
\usepackage{booktabs}   % Better looking tables

% ---------------------------------------------------
% Graphics and Figures
% ---------------------------------------------------
\usepackage{graphicx}   % Include graphics
\usepackage{caption}    % Customization of captions

% ---------------------------------------------------
% Text Styling
% ---------------------------------------------------
\usepackage{color}      % Color support
\usepackage{soul}       % Highlighting, underlining, strikethrough, etc.
\usepackage{parskip}    % Skips line in paragraph instead of indent
% ---------------------------------------------------
% Headers, Footers, and Symbols
% ---------------------------------------------------
\usepackage{fancyhdr}   % Custom headers/footers
\usepackage{lastpage}   % Reference last page in document
\usepackage{bbding}     % Special symbols (checkmarks, etc.)
\usepackage{pmboxdraw}  % Box drawing characters
\usepackage{fontawesome} % Allows customization of icons 

% ---------------------------------------------------
% Section Formatting
% ---------------------------------------------------
\usepackage{titlesec}   % Customize section titles

% ---------------------------------------------------
% Hyperlinks and URLs
% ---------------------------------------------------
\usepackage{url}        % Simple URL typesetting
\usepackage{hyperref}   % Clickable hyperlinks

% ---------------------------------------------------
% FancyHDR Options
% ---------------------------------------------------
\fancyhead[L]{\fontfamily{lmss}\selectfont EC 390}
\fancyhead[R]{\fontfamily{lmss}\selectfont Final Project Part III}
\setlength{\headheight}{13.59999pt} % Fixes caution produced by package


\titleformat{\section}
  {\normalfont\large\bfseries\sffamily}{\thesection}{1em}{}

\titleformat{\subsection}
  {\normalfont\normalsize\bfseries\sffamily}{\thesection}{1em}{}
  
\hypersetup{
    colorlinks=true,
    linkcolor=blue,
    filecolor=magenta,      
    urlcolor=blue,
    citecolor=blue,
}

\pagenumbering{arabic}

\begin{document}

\thispagestyle{fancy}
\fontfamily{lmss}\selectfont

\begin{center}
\textbf{\huge EC 390: Presentation Outline (Rough Draft)}\\  
\end{center}

%%%%%%%%%%%%%%%%%%%%%%%%%%%%%%%%%%%%%%%%%%%%%%%%%%%%%%%%%%%
% ASSIGNMENT OVERVIEW
%%%%%%%%%%%%%%%%%%%%%%%%%%%%%%%%%%%%%%%%%%%%%%%%%%%%%%%%%%%
\section*{Overview}

The third step of the final project involves submitting an outline or rough draft of your presentation.
The purpose is for you to see tangible progress on your term-long project. 
Presenting a project is not an easy task, and usually requires plenty of drafting, editing, and feedback.
The most difficult part of presenting work/projects/proposals is finding a good flow and way to talk about the work. 
There is only so much we can say, while making sense, in 10 minutes. 
The goal of a presentation is to develop those skills. 

%%%%%%%%%%%%%%%%%%%%%%%%%%%%%%%%%%%%%%%%%%%%%%%%%%%%%%%%%%%
% INSTRUCTIONS
%%%%%%%%%%%%%%%%%%%%%%%%%%%%%%%%%%%%%%%%%%%%%%%%%%%%%%%%%%%
\section*{Instructions}

\begin{itemize}
    \item[\faCheckSquareO] \textbf{Prepare a draft presentation} (PowerPoint, Google Slides, or PDF Export)
    \item[\faCheckSquareO] It should include \textbf{all major sections} of your final presentation 
    \begin{itemize}
        \item[\faFolderOpenO] Title slide (with name, project title, and course)
        \item[\faFolderOpenO] Introduction/Motivation: Explain why your topic is important 
        \item[\faFolderOpenO] Research Question: Clearly state the central question or hypothesis 
        \item[\faFolderOpenO] Literature Review: Introduce the research articles you read. Highlight how your question builds on or differs from existing work. 
        \item[\faFolderOpenO] Proposed Approach: Describe how you plan to answer your question. 
        \item[\faFolderOpenO] Expected Contribution: Explain what we might learn from your project and why it matters  
    \end{itemize}
\end{itemize}

%%%%%%%%%%%%%%%%%%%%%%%%%%%%%%%%%%%%%%%%%%%%%%%%%%%%%%%%%%%
% SUBMISSION DETAILS
%%%%%%%%%%%%%%%%%%%%%%%%%%%%%%%%%%%%%%%%%%%%%%%%%%%%%%%%%%%
\section*{Formatting and Submission}

\begin{itemize}
    \item[\faFileO] Length: Equivalent to a \textbf{10-minute presentation} (Roughly $\sim 8$ slides)
    \item[\faFileO] Submit as a \textbf{PDF} or link to an accessible online presentation (if using Google Slides)
    \item[\faFileO] Due \textbf{November 19 at 11:59pm} on Canvas \href{https://canvas.uoregon.edu/courses/274682/assignments/1916915}{(Submit Here)}
\end{itemize}

%%%%%%%%%%%%%%%%%%%%%%%%%%%%%%%%%%%%%%%%%%%%%%%%%%%%%%%%%%%
% GRADING RUBRIC
%%%%%%%%%%%%%%%%%%%%%%%%%%%%%%%%%%%%%%%%%%%%%%%%%%%%%%%%%%%
\section*{Grading}

This draft is graded on \textbf{completion and effort} rather than polish. 
To earn full credit:
\begin{itemize}
    \item[\faTags] All required sections are present and sufficiently developed
    \item[\faTags] Slides show meaningful progress toward a coherent proposal
\end{itemize}

\section*{Tips}

\begin{itemize}
    \item[\faStickyNoteO] Use concise bullet points and clear visuals
    \item[\faStickyNoteO] Focus on making your research question and approach \textbf{understandable to a general economics audience}
    \item[\faStickyNoteO] Clarity and structure matter more than perfection    
\end{itemize}

\end{document}